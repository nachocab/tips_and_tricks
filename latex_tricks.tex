% n choose m
\[
\cdot{365 \choose 15}
\]

% big parenthesis or brackets
\[
\left( X \right)
\]

% union ∪
$\cup$

% intersection ∩
$\cap$

% limit as N goes to infinity
$\lim_{N \to +\infty}$


% plus or minux
\pm

% quotation marks quotes
``blah blah''

% space in math mode
\text{ blah blah}

% listing code
\begin{lstlisting}

\end{lstlisting}

% ⟨i⟩ left and right angle brackets
\langle i \rangle

% asterisk in math mode
$\ast$

% sumatorio
$\sum_{i=1}^t$

% partial derivative
$\partial$

% ô hat circumflex
\hat{o}

% tilde
\tilde{a} \acute{a} \bar{a} \dot{a} \breve{a} \check{a} \grave{a} \vec{a} \ddot{a}          

% literal text, don't interpret, allow all spaces and indentations
\begin{verbatim}
    
\end{verbatim}

% insert image
\begin{figure}[h!]
      \centering
      \includegraphics[width=1\textwidth]{biogenesis.png}
      \caption{miRNA biogenesis}
      % \label{fig:biogenesis}
\end{figure}

% Figure positioning
h   Place the float here, i.e., approximately at the same point it occurs in the source text (however, not exactly at the spot)
t   Position at the top of the page.
b   Position at the bottom of the page.
p   Put on a special page for floats only.
!   Override internal parameters LaTeX uses for determining "good" float positions.
H   Places the float at precisely the location in the LaTeX code. Requires the float package,[1] e.g., \usepackage{float}. This is somewhat equivalent to h!.

% Less than or equal to ≤
\leq

% comment or annotate an equation
\intertext{$u=S(t); du=S'(t)dt$}

% matrix
\begin{align*}
    begin{pmatrix}
        1 & 0\\
        -1 & 0\\
        0 & 1 \\
        0 & -1 \\
    \end{pmatrix}
\end{align*}
% square root cubic
$\sqrt[3]{x+y}$